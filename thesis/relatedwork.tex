\section{Related Work}

% Beginning of abstract interpretation who then expanded on it, what was added
%
% about using proof assistants to do abstract interpretation
Proof assistants can be used to mechanize the proof of abstract interpreters.
However, as this can be tedious work, much effort has been put into easing the
proof burden. Darais et al~\cite{darais2017abstracting} have focused on
abstracting definitional interpreters in order to increase the modularity of
the abstract interpreters. Definitional interpreters leave certain aspects of
the computations implicit by defining a new language that describes the
language that is to be analyzed~\cite{reynolds1972definitional}. Unlike in our
work, where there is a heavy emphasis on continuations through the use of our
AbstractState and State monads, the abstract definitional interpreter
constructed by Darais has no explicit representation of continuations.

One of the problems Darais had to work around was non-termination of the
definitional interpreter. To counteract this required extra work on his part.
Because Coq does not allow writing non-terminating functions, we had no such
problems.

Keidel et al.~\cite{keidel2018compositional} instead propose a way to split up
the concrete and abstract interpreters by defining a shared structure. By
implementing the concrete and abstract interpreters as instances of a shared
intrepreter, the soundness proof of the abstract interpreter can be decomposed
into smaller lemmas, each focusing on how the two instances differ from the
shared part. It is this paper that serves as the basis for this thesis. However,
the proofs in Keidels work are pen-and-paper proofs. By implenting the ideas
presented in Coq instead of Haskell, we can mechanize these pen-and-paper
proofs, making abstract interpreters that are proven sound more accessible.

One large difference between Keidel's paper and this thesis lies in that we use
Monads instead of Arrows to represent effects such as stores and exceptions. As
arrows are generalizations of monads~\cite{hughes2000generalising}, we could
rewrite our implementation to use arrows instead, but we have not seen the need
to do so (yet).
