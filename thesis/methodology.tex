\section{Methodology}
We have decided on an incremental approach to developing the framework. We will
start out by defining the requisite mathematical constructs, such as the
orderings and Galois connections, in Coq. Building on these definitions, we
should then be able to define what it means for an abstract interpreter to be
sound, and provide a rudimentary implementation of a concrete and abstract
interpreter for a toy language.

After defining these basics steps, we will then improve upon it by incorporating
the design of~\cite{keidel2018compositional}, extracting a shared interpreter
from the concrete and abstract interpreters. By proving lemmas about the
shared interpreter and the concrete and abstract implementations, it is hoped
that the resulting proof would be easier to develop than a comparative proof
written for an abstract and concrete interpreter that are not so similarly
structured.

The end result is intended to be a framework to ease the development of formal
proofs of the soundness of abstract interpreters in Coq. It should be possible
for the developers of those proof to define their concrete and abstract
interpreters and receive a series of lemmas to prove. The soundness of the
interpreters should follow from those for free 
by~\cite{keidel2018compositional}.

After the framework has been written, I will verify if the usage of the
framework will indeed result in easier proofs by performing a case study on an
abstract interpreter.

\subsection{Coq}
The work of Coquand et al~\cite{coquand1986calculus} eventually led to the
development of Coq, an interactive proof assistant. Coq implements the calculus
of constructions, a system for constructive proofs. By using proof assistants 
we can mechanically prove the soundness of analyzers. 
However, this is still not often done. It is difficult to
pinpoint the exact reason for this, but we will proceed under the assumption
that the required effort remains to large for the perceived gains. 

We have used Coq to write our definitions and
prove the soundness of the abstract interpreter. ~\cite{coqintroduction}.
In this section, we will give some examples of Coq code along with an
explanation of what is meant by the code. Readers familiar with Coq can skip
this section. Other readers may wish to read it to help understand later code
listings.
\todoin{coq examples}

