\section{Methodology}
We have decided on an incremental approach to developing the framework. We will
start out by defining the requisite mathematical constructs, such as the
orderings and Galois connections, in Coq. Building on these definitions, we
should then be able to define what it means for an abstract interpreter to be
sound, and provide a rudimentary implementation of a concrete and abstract
interpreter for a toy language.

After defining these basics steps, we will then improve upon it by incorporating
the design of~\cite{keidel2018compositional}, extracting a shared interpreter
from the concrete and abstract interpreters. By proving lemmas about the
shared interpreter and the concrete and abstract implementations, it is hoped
that the resulting proof would be easier to develop than a comparative proof
written for an abstract and concrete interpreter that are not so similarly
structured.

The end result is intended to be a framework to ease the development of formal
proofs of the soundness of abstract interpreters in Coq. It should be possible
for the developers of those proof to define their concrete and abstract
interpreters and receive a series of lemmas to prove. The soundness of the
interpreters should follow from those for free 
by~\cite{keidel2018compositional}.

After the framework has been written, I will verify if the usage of the
framework will indeed result in easier proofs by performing a case study on an
abstract interpreter.

\subsection{Coq}
The work of Coquand et al~\cite{coquand1986calculus} eventually led to the
development of Coq, an interactive proof assistant. Coq implements the calculus
of constructions, a system for constructive proofs. By using proof assistants 
we can mechanically prove the soundness of analyzers. 
However, this is still not often done. It is difficult to
pinpoint the exact reason for this, but we will proceed under the assumption
that the required effort remains to large for the perceived gains. 

We have used Coq to write our definitions and
prove the soundness of the abstract interpreter. ~\cite{coqintroduction}.
In this section, we will give some examples of Coq code along with an
explanation of what is meant by the code. Readers familiar with Coq can skip
this section. Other readers may wish to read it to help understand later code
listings.

\begin{minted}{coq}
Inductive avalue_int : Type := 
  | VInt : interval → avalue_int
  | VAbool : abstr_bool → avalue_int
  | VTop : avalue_int.
\end{minted}
In Coq, we define new types using the Inductive command. In the above example,
we define a new type \coq{avalue_int } with three constructors, \coq{VInt },
\coq{VAbool } and \coq{VTop }. A constructor is a function that can create the
inductive type. The \coq{VInt } and \coq{VAbool } constructors each take one
argument and \coq{VTop } doesn't take any.

\begin{minted}{coq}
Definition MaybeT (M : Type → Type) (A : Type) : Type := M (Maybe A).

Definition parity_plus (p q : parity) : parity :=
  match p, q with 
  | par_top, _ | _, par_top => par_top
  | par_even, par_even | par_odd, par_odd  => par_even
  | par_odd, par_even  | par_even, par_odd => par_odd
  end.
\end{minted}

The Definition command creates type aliases or functions. In the first case, we
bind the identifier \coq{MaybeT M A } to \coq{M (Maybe A) }. The definition
also requires that the parameters \coq{M } and \coq{A } be of types \coq{Type
-> Type} and \coq{Type} respectively.

The second Definition defines a function \coq{parity_plus } that takes 
two parities \coq{p } and \coq{q } and returns a new parity. Via the match
tactic, Coq looks at which constructors where used to build \coq{p } and
\coq{q } and returns the value associated with that case. The pipe symbol $|$
precedes a set of constructors for \coq{p } and \coq{q } and can be used to
group cases together that have the same output. The underscore indicates that
we are not interested in the specific value. Together, this means that the
first case \coq{ | par_top, _ | _, par_top => par_top } matches any combination
of values for \coq{p } and \coq{q } where either one of them is \coq{par_top }.

\begin{minted}{coq}
Lemma parity_plus_par_even : forall p,
  parity_plus p par_even = p.
Proof. intros. destruct p; reflexivity. Qed.
\end{minted}

The value of Coq comes from the ability to take those definitions and proof
properties about them. We start a proof with the \coq{Lemma } command. In the
above example, we define a lemma called \coq{parity_plus_par_even} which
states that for all parities, adding a value of \coq{par_even } does not change
the original value. Proofs are delimited by the \coq{Proof.} and \coq{Qed.} 
commands. In between these commands come a series of \emph{tactics}. 

\begin{minted}{coq}
Class Galois (A A' : Type) {A'_preorder: PreorderedSet A'} : Type :=
{
  gamma : A' -> A -> Prop;
  gamma_monotone : monotone gamma;
}.
\end{minted}

In this thesis we utilize the typeclass mechanism of Coq. Typeclasses are a way
to define methods and lemmas for types that implement that typeclass. In the
above example, we define a Galois typeclass that takes two types, and a proof
that the second type \coq{A'} forms a preordered set. It further states that
implementation of this typeclass should define a function \coq{gamma} of type
\coq{A' -> A -> Prop} and a proof that this function gamma is monotone.

\begin{minted}{coq}
Inductive gamma_par : parity → nat → Prop :=
  | gamma_par_even : ∀ n, Nat.Even n → gamma_par par_even n
  | gamma_par_odd  : ∀ n, Nat.Odd n → gamma_par par_odd n
  | gamma_par_top  : ∀ n, gamma_par par_top n.
Hint Constructors gamma_par : soundness.

Lemma gamma_par_monotone : monotone gamma_par.
Proof. (* proof omitted *) Qed.

Instance galois_parity_nat : Galois nat parity :=
{
  gamma := gamma_par;
  gamma_monotone := gamma_par_monotone;
}.
\end{minted}

To create an instance of a typeclass, we need to define the required functions
and lemma's. Here, we have defined an inductive relation called \coq{gamma_par}
and proven it to be monotone. We can then use those definition in creating an
\coq{Instance} of the Galois typeclass.

