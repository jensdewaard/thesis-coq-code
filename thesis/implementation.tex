\section{Implementation}
In this section we will describe how the framework was implemented in Coq and
the theory behind the implementation. First, we will delve into the theory
behind monads~\cite{wadler1995monads}. We borrow these constructs from the 
Haskell Prelude and reimplement them in Coq. Secondly, we shall describe the 
syntax of a toy language and define a concrete interpreter of this language. We
will show how we can decompose the concrete interpreter due to its monadic
structure. After that, we
dive into the theory behind Galois connections. After defining Galois
connections, we show how there is a Galois connection between the previously
defined concrete interpreter and an example of an abstract interpreter.
Finally, we show what it means for such an abstract interpreter to be sound.

\subsection{Monads}
Many tutorials explaining monads exist and monads are generally regarded as
difficult to understand. We will briefly try to explain them here, though
a complete understanding monads is not neccessary to follow later sections.
Understanding that we can decompose the program along the lines of the methods
provided by the Monad typeclass is sufficient.

Monads are a way to add impure effects to a pure language. Examples of impure
effects are modifying a global state, throwing and handlings exceptions and
perform IO operations. It is often said that monads represent
\textit{computations}. 

In Coq we can implement monads in the form of a typeclass, as it is done in
Haskell~\cite{sozeau2008first}. Typeclasses are somewhat like interfaces in 
object-oriented programming languages. A programmer can write functions 
operating on typeclasses instead of concrete types, and those functions can
accept all types that implement the functions defined in that specific typeclass. 
In this case, all types belonging to the Monad typeclass must implement two functions,
\textit{bind}, originally called \textit{*} in the paper by Wadler, and 
\textit{return}, originally called \textit{unit}. An implementation of monads
as a typeclass in Coq is given in Listing \ref{lst:sample_monad}.

\begin{listing}[H]
\begin{minted}{Coq}
Class Monad (M : Type -> Type) : Type :=
{
  returnM : forall {A}, A -> M A;
  bindM : forall {A B}, M A -> (A -> M B) -> M B;
}.
Notation "x >>= y" := (bindM x y) (at level 40, left associativity).
\end{minted}
\caption{Example of an implementation of Monad as a Coq typeclass}
\label{lst:sample_monad}
\end{listing}

Because return is a reserved keyword in Coq, we rename both \textit{return} and 
\textit{bind} into \textit{returnM} and \textit{bindM} respectively.
\textit{returnM} takes a value and \"wraps\" it in the monad. \textit{bindM} is
a function that can apply a function to a monadic value. As is the convention
in Haskell, we will use $>>=$ as an infix notation for the bindM
function.

One of the advantages that Coq typeclasses have over those in Haskell is the
abilty to add lemmas to the typeclass in addition to the methods. Proper monads
have to observe three laws, together called the Monad laws. These laws are, as
written in the original paper
\begin{align*}
    \textit{unit}\ a >>= \lambda b. n =&\ n[a/b] \\
    m >>= \lambda a . \textit{unit}\ a =&\ m \\
    m >>= (\lambda a.n >>= \lambda b.o) =&\ (m >>= \lambda a.n) >>= \lambda b.o
\end{align*}

In English, these mean that binding \textit{returnM} to a function f is
equivalent to just applying $f$ to the value wrapped by \textit{returnM}. The
second law states that binding to \textit{returnM} is effectively a no-op, and
the last law requires \textit{bindM} to be associative. In Coq, we can add the
following lemmas to the Monad typeclass to ensure that all typeclass instances
are valid monads, which is something that Haskell cannot guarantee. 

\begin{listing}[H]
\begin{minted}{Coq}
Class Monad (M : Type -> Type) : Type := {
  bind_id_left : forall {A B} (f : A -> M B) (a : A), 
    bindM (returnM a) f = f a;
  bind_id_right : forall {A} (MA : M A),
    bindM MA returnM = MA;
  bind_assoc : forall {A B C} (MA : M A) (f : A -> M B) (g : B -> M C),
    bindM (bindM MA f) g = bindM MA (λ a, bindM (f a) g);
}
\end{minted}
\caption{Lemmas for the monad laws}
\label{lst:monad_lemmas}
\end{listing}

In this thesis, we consider monads that add two different side effects. The
first is the Maybe monad, also known as Option. The Maybe monad adds the
possibility of failure to an operation. The other is the State monad, which
allows us to keep track of a global state. In the next few sections we will
discuss the implementation of these monads.

\subsection{Maybe}
\begin{listing}[H]
\begin{minted}{coq}
Inductive Maybe A : Type :=
    | Just : A -> Maybe A
    | None : Maybe A.
\end{minted}
\caption{The Maybe monad}
\label{lst:maybe_monad}
\end{listing}

The first monad we implemented was the Maybe monad. Values of the type
\mintinline{coq}{Maybe A } can be of either the form 
\mintinline{coq}{Just a }, where $a$ is a value of type $A$ or 
\mintinline{coq}{None }. Here, \mintinline{latex}{None} represent a failed
computation. For example, a monadic version of division \mintinline{coq}{nat ->
nat -> Maybe nat } could return \mintinline{coq}{None } when dividing by zero,
and a \mintinline{coq}{Just } value with the right answer in other cases.

The return method of Maybe is the same as the Just constructor. 
The bind method of the Maybe monad performs a case analysis on the provided
value. In the case of Just, it extracts the inner value and applies the given
function to it. In the case of None, it returns None. This allows the chaining
of multiple operations without explicit error handling between each step.

\begin{listing}[H]
\begin{minted}{coq}
  Definition bind_maybe {A B} 
  (m : Maybe A) (k : A -> Maybe B) : Maybe B :=
    match m with
    | None => None
    | Just a => k a
    end.
\end{minted}
\caption{The bind method for Maybe}
\label{lst:bind_maybe}
\end{listing}

\subsection{State}
When we evaluate a program, we will require a way to keep track of a global
state. This state can be as simple as an integer that keeps a count of the
number of operations performed. In our case, we will use a map of strings to
values $V$,  modelling the use of variables in our program. The specific type 
of V will be described in a later section.

The State monad allows us to add the ability for tracking state to a pure
program. It is defined as 
\mintinline{coq}{Definition State (S A : Type) := S -> (A * S). } The State
monad wraps a value in a function that keeps track of an additional value of
type S. Looking at the types of State and return shows us how we ought to
implement the return function for State. When using return, the type for S 
is automatically inferred from the context.

\begin{listing}[H]
\begin{minted}{coq}
Definition return_state := fun a : A => fun st : S => (a, st).
\end{minted}
\end{listing}

The bind method for State passes the state variable along to the next
computation. Updates states are passed along and the current state is
accessible at each step of the computation.operation.

\begin{listing}[H]
\begin{minted}{coq}
  Definition bind_state {A B} 
    (p : State S A) (k : A -> State S B) : State S B :=
    fun st => match (p st) with
              | (x, st') => k x st'
              end.
\end{minted}
\caption{The implementation of bind for the State monad}
\label{lst:bind_state}
\end{listing}

Listing~\ref{lst:bind_state} shows us how the bind method is implemented. The
result of binding a State value to a function is a new function (as all values
of State need to be) in which the provided state variable is applied to the
State value. The result of this computation, including the new updated state,
is then passed to the continuation function $k$.

The State and Maybe monads show us how we would implement the desired
properties of our Toy language in a pure manner. But right now, we will have to
change between the two. We can either have the evaluator utilize the State
monad to keep track of state, or have it use the Maybe monad to allow
exceptions, but we cannot have a runtime in which we can both throw exceptions
as well as keep track of a global state. 

There is a way to combine the functionalities of monads via the use of monad
transformers. 

\subsection{Monad Transformers}
Monad transformers are a way to add the side effects of a monad to another
monad. To facilitate this, we introduce another typeclass.

\begin{listing}
\begin{minted}{coq}
  Class MonadT {M} `{Monad M} (T : (Type -> Type) -> Type -> Type)  `{Monad (T M)} : Type :=
  {
    liftT : forall {A}, M A -> T M A;
    lift_return : forall  {A} (x : A), liftT (returnM x) = returnM x;
    lift_bind : forall {A B} (x : M A) (f : A -> M B),
      liftT (x >>= f) = liftT x >>= (f ∘ liftT);
  }.
\end{minted}
\end{listing}

The notation \textit{`Monad (T M)} indicates that whenever we define an
typeclass instance for a type T, we require that (T M), which should be of type
Type -> Type, is also a monad instance. That means that all the returnM and
bindM methods, as well as the monad laws, apply to the transformed monad (T M).
In addition, we also have access to a new method liftT, which allows us to
utilize methods defined for the underlying monad by "lifting" them to the
transformed monad. 

We define two transformer monads that correspond to the monads defined above:
MaybeT and StateT. These extend a monad by allowing for failing computations
and the ability to retain a global state. 

\begin{listing}
\begin{minted}{coq}
Definition MaybeT M A: Type := M (Maybe A).
Definition StateT S M A : Type := S → M (A*S)%type.
\end{minted}
\caption{Definitions for the monad transformers}
\end{listing}

\subsection{The Concrete Interpreter}

\subsubsection{Language Syntax}\label{sec:lang}
Programs consist of statements and expressions. These programs are evaluated by
interpreters. In this subsection, we shall describe the syntax of a toy
language.


\begin{minted}{Coq}
Inductive cvalue : Type :=
  | VNat : nat -> cvalue
  | VBool : bool -> cvalue.
\end{minted}

Our toy language has two types of values. Natural numbers, represented by
\coq{VNat } and boolean values, represented by \coq{VBool }. 
When a \coq{cvalue }, short for concrete value is needed, any of these two 
types can be used.

Expressions are operations on values that result in a new value. They are
implemented as an inductive type called \coq{expr}. 

\begin{minted}{Coq}
Inductive expr : Type :=
  | EVal : cvalue -> expr
  | EVar : string -> expr
  | EPlus : expr -> expr -> expr
  | EMult : expr -> expr -> expr
  | EEq : expr -> expr -> expr
  | ELe : expr -> expr -> expr
  | ENot : expr -> expr
  | EAnd : expr -> expr -> expr.
\end{minted}

Expressions can exist of concrete values (`5', `true'), names of variables
saved in a store (see Section~\ref{sec:stores}) and operations on these values
(+, *, ==, $<$, negation and $\&\&$). 

\begin{minted}{Coq}
Inductive com : Type :=
  | CSkip : com
  | CSeq : com -> com -> com
  | CAss : string -> expr -> com
  | CIf  : expr -> com -> com -> com
  | CTryCatch : com -> com -> com
  | CThrow : com. 
\end{minted}

Statements control the flow of the program. In this case, we have implemented a
skip statements that effectively does nothing, an assigment statement that
assigns the result of an expression to a variable, an if statement, a try-catch
statement and a throw statement that acts as an exception.

\subsubsection{Stores}\label{sec:stores}
When evaluating the assignment statement \coq{CAss} and the variable
expression \coq{EVar}, the interpreter should write to and read from a variable
storage. We call these stores. 

The concrete store is a map\footnote{The \textit{total\_map} comes from 
	Map.v from the public Software Foundations course.} 
from strings to cvalues. The store is implemented as a total map, which is a
map that returns a default value when a key is requested that is not present in
the map. Values can be added to the map via the \coq{t_update} function.

% maybe explain why I didn't use a partial map?
\begin{minted}{Coq}
Definition store := total_map cvalue.

Definition t_update {A:Type} (m : total_map A)
                    (x : string) (v : A) :=
  fun x' => if beq_string x x' then v else m x'.

\end{minted}

\subsubsection{Monads}\label{sec:monad}
Monads are a design pattern used to represent computation. Much has been
written about how they work, and monads are in fact notorious for being
difficult to understand. We will not go into detail here, and instead just 
provide the definition of the monad typeclass and an implementation of
it~\cite{kleisli1965every}.

\begin{minted}{Coq}
Class Monad (M : Type -> Type) : Type :=
{
  returnM : forall A, A -> M A;
  bind : forall A B, M A  -> (A -> M B) -> M B;
}.
\end{minted}

\begin{minted}{Coq}
Definition State (A : Type) := store -> result A store.

Definition return_state (A : Type) (x : A) : State A :=
  fun st => returnR A store x st.

Definition bind_state (A B : Type) (m : State A) (f : A -> State B) 
    : State B :=
  fun st => match m st with
            | returnR _ _ x st' => f x st'
            | crashed _ _ => crashed _ _
            | exception _ _ st' => exception _ _ st'
            end.

Instance state_monad : Monad (State) := {
  returnM := (return_state);
  bindM := (bind_state);
}.
\end{minted}

Monads require two functions, a \coq{return} function that wraps a value as a
monadic value and a \coq{bind} function that is able to chain operations on the
monadic values. 

The State monad is an instance of a monad that we can use to propagate state
changes when analyzing programs. We are able to abstract away how data is
stored and instead focus on the transformations on values.

\subsubsection{Do notation}
The interpreters make excessive use of the bind function of the State monad
defined in Section~\ref{sec:monad}. Because of this, we introduce a special
notation for the function.

\begin{minted}{Coq}
Notation "x '<<' y ; z" := (bind y (fun x => z))
  (at level 20, y at level 100, 
   z at level 200, only parsing).
\end{minted}

If we did not include this notation, making use of the Monad would quickly
become cumbersome. With it, we are able to write data pipelines in an
imperative style. To illustrate, the below code shows the two
different styles. As the length of the `imperative' program increases, the
functional notation grows more cumbersome and error prone.

\begin{minted}{Coq}
ENot e =>
	v << (eval_expr e) ;
	b << (ensure_bool v) ;
	returnM (VBool (negb b))

ENot e =>
	bind (eval_expr e) (fun v =>
		bind (ensure_bool v) (fun b =>
			returnM (VBool (negb b))))
\end{minted}


\subsubsection{The interpreter}
As shown in Section~\ref{sec:lang} the syntax of the language allows the two
types of values, booleans and natural numbers, to be used interchangably while
still have a valid syntax. However, most operations will be meaningless if used
on the wrong operators. For example, ``1 + true'' is valid syntax, but it is 
not a valid program.

To ensure that the correct types are used when evaluating operators,
the interpreter has an `ensure' function for each of the types. 

\begin{minted}{Coq}
Definition ensure_nat (v : cvalue) : State nat :=
  fun st => match v with
            | VNat x => returnR nat store x st
            | _ => crashed _ _
            end.
            
Definition ensure_bool (v : cvalue) : State bool :=
  fun st => match v with
            | VBool b => returnR bool store b st
            | _ => crashed _ _
            end.
\end{minted}

When the ensure function is called with a type other than the desired type, it
returns a crashed state. This crashed state is unrecoverable, meaning that the
interpreter cannot return to a valid state without intervention from the
programmer. The crashed state is propagated throughout the rest of the program.

\begin{minted}{Coq}
Fixpoint eval_expr (e : expr) : State cvalue :=
  match e with
  | EVal x => returnM x
  | EVar x => st << get ;
      returnM (st x)
  | EPlus e1 e2 => 
      v1 << (eval_expr e1) ;
      v2 << (eval_expr e2) ;
      n1 << (ensure_nat v1) ;
      n2 << (ensure_nat v2) ;
      returnM (VNat (n1 + n2))
  | EMult e1 e2 =>
      v1 << (eval_expr e1) ;
      v2 << (eval_expr e2) ;
      n1 << (ensure_nat v1) ;
      n2 << (ensure_nat v2) ;
      returnM (VNat (n1 * n2))
  | EEq e1 e2 =>
      v1 << (eval_expr e1) ;
      v2 << (eval_expr e2) ;
      n1 << (ensure_nat v1) ;
      n2 << (ensure_nat v2) ;
      returnM (VBool (Nat.eqb n1 n2))
  | ELe e1 e2 =>
      v1 << (eval_expr e1) ;
      v2 << (eval_expr e2) ;
      n1 << (ensure_nat v1) ;
      n2 << (ensure_nat v2) ;
      returnM (VBool (Nat.leb n1 n2))
  | ENot e =>
      v << (eval_expr e) ;
      b << (ensure_bool v) ;
      returnM (VBool (negb b))
  | EAnd e1 e2 =>
      v1 << (eval_expr e1) ;
      v2 << (eval_expr e2) ;
      b1 << (ensure_bool v1) ;
      b2 << (ensure_bool v2) ;
      returnM (VBool (andb b1 b2))
  end.
\end{minted}

The concrete interpeter follows the same structure for each kind of operator.
The smaller expressions are evaluated first via recursive calls to the
evaluation function and the interpreter
ensures they are of the proper types. Then, the built-in operator corresponding
to the expression is called on those values.

\begin{minted}{Coq}
Definition eval_if {A} (b : bool) (st1 st2 : State A) : State A := 
  if b then st1 else st2.

Definition eval_catch {A} (st1 st2 : State A) : State A :=
  fun st => match (st1 st) with
  | crashed _ _ => crashed _ _
  | exception _ _ st' => (st2 st')
  | x => x
  end.
\end{minted}
To ease the comparison between the concrete interpreter and abstract
interpreters later on, we have factored out the evaluation of the if and catch
expressions. When these have been factored out, proving the soundness of these
operators can be put into several smaller lemmas, as can be seen in Section
\ref{sec:soundness}.

The concrete interpretation of the If statement is simply a wrapper around the
built-in if construct of Coq. 

The evaluation of the try-catch statement is a
match function of the result of the first statement block. If the first block
crashes, the result of the entire try catch statement is also a crash. Recall
that crashes should only occur in the case of code errors, by writing
a program that for example adds a boolean to an integer.

If the first block raises an exception through the use of the fail statement,
the interpreter returns the result of the evaluation of the second block. In
any other case, which is just the succesful evaluation of the first block, the
second block is disregarded and the result of the evaluation of the first block
is used.

\begin{minted}{Coq}
Fixpoint ceval (c : com) : State unit :=
  match c with
  | CSkip => returnM tt
  | c1 ;c; c2 => 
      (ceval c1) ;; (ceval c2)
  | x ::= a => 
      n << (eval_expr a) ;
      st << get ;
      put (t_update st x n)
  | CIf b c1 c2 => 
      v << (eval_expr b) ;
      b' << (ensure_bool v) ;
      eval_if b' (ceval c1) (ceval c2)
  | try c1 catch c2 =>
   	  eval_catch (ceval c1) (ceval c2)
  | CFail => fail
  end.
\end{minted}

These functions are used in the concrete interpeter of the language statements.
The `ceval' function uses many of the earlier defined functions and operations,
such as retrieving and inserting values from/to maps and the monadic operations
defined in Section~\ref{sec:monad}.

\subsection{Galois Connections}
In this section we will describe the notion of Galois definitions. Before we
can describe those however, we will have to explain some other concepts. We
start with typeclasses, constructs that first appeared in the Haskell
programming language~\cite{wadler1989make}. We then define type classes for
preordered sets, before moving on to define a type class for Galois
connections.

\subsubsection{Type Classes}
In order to be able to automate the soundness proofs as much as possible, we
will need to define several mathematical concepts in Coq. In this section, we
will describe those concepts and show how they are implemented in the form of
Coq typeclasses~\cite{sozeau2008first}. 

Type classes are a method to add constraints to the parameters of functions.
Consider the type class \coq{Eq} in the code below.

\begin{minted}{Coq}
Class Eq (X : Type) : Type :=
{
	eq_op : X -> X -> Prop;
}.
\end{minted}

In plain language, this would mean that all types \coq{X}, on which there is a
function \coq{eq_op} defined with the proper signature, belong to the type
class \coq{Eq}. 

To show that a type belongs to the type class, we have to provide what is
called an instance.

\begin{minted}{Coq}
Instance nat_eq : Eq Nat :=
{
	eq_op := Nat.eq;
}.
\end{minted}

In the above snippet, we prove that the built-in \coq{Nat} type is part of
the \coq{Eq} type class, with the equality operator on natural numbers as the
\coq{eq_op} function.

\subsubsection{Preordered sets}
The first definition we shall implenent is the notion of a preordered set. 
A preordered set is any set S on which we can define a preorder relation
$\sqsubseteq$ such that the preorder relation is reflexive and transitive. That
is, for all $a, a \sqsubseteq a$ (reflexivity) and for all $a$ $b$ $c$, if $a
\sqsubseteq b$ and $b \sqsubseteq c$, then $a \sqsubseteq c$ (transitivity).
We refer to such a combination of set S and relation as $(A, \sqsubseteq)$.

The definitions of reflexive and transitive relations are already included in
the standard library of Coq, so we can import those. This gives us the
following definition of the preorderd set as a Coq type class.

\begin{minted}{Coq}
Class PreorderedSet (X : Type) : Type :=
{
  preorder : X -> X -> Prop;
  preorder_refl: Reflexive preorder;
  preorder_trans: Transitive preorder;
}.
\end{minted}

Defining an instance of this typeclass requires providing a type and a preorder
relation, in addition to proofs of the reflexive and transitive properties of
that preorder. 

\subsubsection{Galois connections}
We require the definition of a preordered set in order to be able to define a
Galois connection.

Traditionally, Galois connections are defined on two partially ordered sets.
However, they can also be defined on preordered sets, and for our purposes that
works too. The difference between preordered sets and partially ordered sets is
that a partial order is antisymmetric in addition to reflexive and transitive.
We do not require the antisymmetric property, so we can loosen that restraint.

The definition of Galois connections that we have implemented is that of the
monotone Galois connection. Given two preordered sets $(A,\sqsubseteq)$ and
$(B,\sqsubseteq)$, a Galois connection between those two sets is defined as two
functions $\alpha : A \rightarrow B$ and $\gamma : B \rightarrow A$ such 
that $\alpha(a)
\sqsubseteq b$ iff $a \sqsubseteq \gamma(b)$.

\begin{minted}{Coq}
Class Galois (A B : Type) `{PreorderedSet B} : Type :=
{
  gamma : B -> A -> Prop;
  gamma_monotone : monotone gamma;
}.
\end{minted}

Instancing the Coq type class definition of Galois connections between two
types $A$ and $B$ requires supplying those types, as well as a gamma function
$gamma : B \rightarrow A \rightarrow Prop$ that is monotone. 
A monotone function $f : A \rightarrow B$ is
a function between two preordered sets $A$ and $B$ such that if $a \sqsubseteq
a'$, then $f(a) \sqsubseteq f(a')$.

Note that the Coq definition does not require A to be a preordered set. This is
because we actuall consider Galois connections between $\mathcal{P}(A)$ and
$B$. For example, we will consider the relation between sets of natural numbers
e.g.\ $\{2, 4, 8\}$ and parities e.g.\ even.

Another difference between the Coq implementation and the mathematical
definition is the lack of corresponding \textit{alpha} function. This is
because we are unable to define this function, due to limitations in Coq.
Luckily, it turns out that the $\alpha$ function is not necessary to prove
soundness~\cite{jourdan2016verasco}.

It is difficult to grasp what a Galois connection means from the definition
alone, so we will provide a few examples in the next section, after
introducing the abstract types.

\subsubsection{Parities}
The first abstract type defined was \coq{Parity}.

\begin{minted}{Coq}
Inductive parity : Type :=
  | par_even : parity
  | par_odd : parity
  | par_top : parity
  | par_bottom : parity.
\end{minted}

The parity of a natural number is whether it is even or odd; e.g.\ the parity
of the number 2 is even, and the parity of the number 9 is odd.

The top value represents any natural number, and the bottom value is the empty
set of numbers. As such, it should not occur during normal execution of the 
program.

We can define a Galois connection between natural numbers and the parity type.
To do so, we must provide a gamma function and prove its monotonicity. Recall
that the signature is of the form $A \rightarrow B \rightarrow Prop$. In this
case, we shall use \coq{parity} as $A$ and \coq{nat} as $B$.

Any type of the form $B \rightarrow Prop$ can be seen as sets of $B$. We define
a function \coq{gamma_par} as taking a parity as a parameter and returning
sets of natural numbers.

\begin{minted}{Coq}
Definition gamma_par (p : parity) : nat -> Prop :=
  match p with
  | par_even => even 
  | par_odd => odd 
  | par_top => (fun n => True)
  | par_bottom => fun n => False
  end.
\end{minted}

In the provided code, the even and odd symbols are imported from the Coq
standard library. They are the sets of even and odd numbers respectively. Each
of the four possible values of parity is mapped to a set of natural numbers.

\begin{minted}{Coq}
Instance galois_parity_nat : Galois nat parity :=
{
  gamma := gamma_par;
  gamma_monotone := gamma_par_monotone;
}.
\end{minted}

The above code shows how a Galois connection between sets of natural numbers 
and parities is instantiated. \coq{gamma_par_monotone} refers to the proof the
the \coq{gamma_par} function is monotone. This proof has been omitted for 
brevity.

\subsubsection{Intervals}
Parities are only one possible abstraction of natural numbers. Another
possibility would be to use intervals. We have not implemented this in our
work, but we will explain how this abstraction would work as another example,
to further clarify the concept of abstractions.

An interval of natural numbers is a valid abstraction of a set of natural
numbers if each of the natural numbers in the set falls within the interval.
For example, the the interval $[1,5]$ is an abstract approximation of the set
$\{2, 4, 5\}$. It is also a valid approximation of the set $\{1,5\}$. This
again illustrates the fact that the abstract approximation often lose 
precision.

The same operators than can be defined on natural numbers can be defined on
intervals. Addition, multiplication and comparisons can all be implemented.
Adding the intervals $[1,3]$ and $[4, 7]$ would result in the interval $[5,
10]$, for example. 

\subsubsection{Abstract Booleans}
The abstract booleans are only slightly different from their concrete
counterpart. We supplement the normal boolean values (true and false) with
values for top and bottom. As with the parities, these abstract values
represent all concrete values and no concrete values respectively.

\begin{minted}{Coq}
Inductive abstr_bool : Type :=
  | ab_true   : abstr_bool
  | ab_false  : abstr_bool
  | ab_top    : abstr_bool
  | ab_bottom : abstr_bool.
\end{minted}

\begin{minted}{Coq}
Definition gamma_bool (ab: abstr_bool) (b : bool) : Prop :=
  match ab with
  | ab_true   => Is_true b
  | ab_false  => ~Is_true b
  | ab_top    => True
  | ab_bottom => False
  end.
\end{minted}

The gamma function for booleans is very straightforward due to the limited
number of concrete values. As with \coq{gamma_par}, the top and bottom
values are mapped to True and False.

\subsection{The Abstract Interpreter}
As with the concrete interpeter, the abstract interpreter has the
\coq{ensure_*} functions for the abstract types. 
They are very similar to the ensure functions above and are omitted for brevity. 

The concrete interpreter used the \coq{State} monad to persist state through
the program. The abstract interpreter has a corresponding monad called the \\
\coq{AbstractState} monad.

\begin{minted}{Coq}
Definition AbstractState (A : Type) :=
  abstract_store -> result A abstract_store.

Definition return_state_abstract (A : Type) (x : A) : AbstractState A :=
  fun st => returnR A abstract_store x st.

Definition bind_state_abstract (A B : Type) 
  (m : AbstractState A) (f : A -> AbstractState B) : AbstractState B :=
  fun st => match m st with
            | returnR _ _ x st' => f x st'
            | crashed _ _ => crashed _ _
            | exception _ _ st' => exception _ _ st' 
            end.

Instance abstract_state_monad : Monad (AbstractState) := {
  returnM := return_state_abstract;
  bindM := bind_state_abstract;
}.
\end{minted}

The abstract state differs from the concrete state in what type of store is
used. The abstract store is almost the same is the concrete store, but returns
abstract values instead of concrete values. When concrete values are inserted
in the store, they are first converted into their abstract counterpart.

\begin{minted}{Coq}
Fixpoint eval_expr_abstract (e : expr) : AbstractState avalue :=
  match e with
  | EVal x => returnM (extract x)
  | EVar x => st << get_abstract ;
      returnM (st x)
  | EPlus e1 e2 => 
      v1 << (eval_expr_abstract e1) ;
      v2 << (eval_expr_abstract e2) ;
      n1 << (ensure_par v1) ;
      n2 << (ensure_par v2) ;
      returnM (VParity (parity_plus n1 n2))
  | EMult e1 e2 =>
      v1 << (eval_expr_abstract e1) ;
      v2 << (eval_expr_abstract e2) ;
      n1 << (ensure_par v1) ;
      n2 << (ensure_par v2) ;
      returnM (VParity (parity_mult n1 n2))
  | EEq e1 e2 =>
      v1 << (eval_expr_abstract e1) ;
      v2 << (eval_expr_abstract e2) ;
      n1 << (ensure_par v1) ;
      n2 << (ensure_par v2) ;
      returnM (VAbstrBool (parity_eq n1 n2))
  | ELe e1 e2 =>
      v1 << (eval_expr_abstract e1) ;
      v2 << (eval_expr_abstract e2) ;
      n1 << (ensure_par v1) ;
      n2 << (ensure_par v2) ;
      returnM (VAbstrBool (ab_top))
  | ENot e =>
      v << (eval_expr_abstract e) ;
      b << (ensure_abool v) ;
      returnM (VAbstrBool (neg_ab b))
  | EAnd e1 e2 =>
      v1 << (eval_expr_abstract e1) ;
      v2 << (eval_expr_abstract e2) ;
      b1 << (ensure_abool v1) ;
      b2 << (ensure_abool v2) ;
      returnM (VAbstrBool (and_ab b1 b2))
  end.
\end{minted}

The abstract interpreter for expressions looks much the same as the concrete
version. The differences are in which type of ensure function is called and in
that the operations on the types are now the abstract versions. For example,
where the concrete interpreter called the standard plus operator for natural
numbers, the abstract interpreter calls the parity plus function.

Notice also that in the case of evaluating a literal value, the abstract
interpreter differs from the concrete version in that the value is wrapped with
an extract function. This function converts a concrete version into its
abstract representation.

\begin{minted}{Coq}
Definition extract (v : cvalue) : avalue :=
  match v with
  | VNat x => VParity (extract_par x)
  | VBool x => VAbstrBool (extract_bool x)
  end.
\end{minted}

\subsection{Extracting the Shared Interpreter}\label{sec:shared_interpreter}
As observed in the previous section, the abstract and concrete interpreter
share many similarities. This leads us to the question of whether we can
decompose the interpreters along the shared structure in a way that eases the
proof burden. In this section, we look at how we've extracted the shared
interpreter and how this shared structure can be instantiated to create
abstract and concrete interpreters like those defined earlier.

To be able to define the shared interpreter, we will need to define interfaces
that can be instantiated as abstract and concrete versions. In Coq, we can
define these interfaces in the form of typeclasses. We shall first discuss the
Store typeclass.

\subsubsection{Stores}
The Store typeclass defines the interface for getting and setting values in the
stores. 
\begin{minted}{Coq}
Class Store (M : Type -> Type) (valType : Type) :=
{
  get : string -> M valType;
  put : string -> valType -> M unit;
}.
\end{minted}

The interface exposes two functions, get and put. We abstract over the specific
monad that is used to pass the state, and we can instantiate the typeclass with
the value type the store is concerned with.

\begin{minted}{Coq}
Definition store_get (x : string) : State cvalue := fun st =>
  returnR (st x) st.

Definition store_put (x : string) (v : cvalue) : State unit := 
  fun st => returnR tt (t_update st x v).

Instance store_concrete : Store State cvalue := {
  get := store_get;
  put := store_put;
}.
\end{minted}

In the above Coq code, we see that the concrete store is instantiated with the
concrete State monad and concrete values. Correspondingly, the abstract store
is instantiated with the AbstractState monad and abstract values.

\subsubsection{Exceptions}
Like with stores, we will require an interface for every side-effect our
programs will have. One of those side-effects is the ability to throw
exceptions. For this, we have the Except typeclass.

\begin{minted}{Coq}
Class Except (M : Type -> Type) := {
  throw    : M unit;
  trycatch : M unit -> M unit -> M unit;
}.
\end{minted}

The Except typeclass has functions for throwing exceptions and a structure for
handling the thrown exceptions. As with the Store interface, it is not
concerned with the specific monad used, which aids in the composability of the
structure. 

\begin{minted}{Coq}
Definition eval_catch_abstract {A} `{Joinable A} 
  (st1 st2 : AbstractState A) : AbstractState A :=
  fun st => match (st1 st) with
  | crashedA => crashedA 
  | exceptionA st' => st2 st'
  | exceptionOrReturn x st' => 
      join_op (exceptionOrReturn x st') (st2 st')
  | returnRA x st' => returnRA x st'
  end.

Definition fail_abstract {A : Type} : AbstractState A :=
  fun st => exceptionA st.

Instance except_abstract : Except AbstractState := 
{
  throw := fail_abstract;
  trycatch := eval_catch_abstract;
}.
\end{minted}

The above listing shows the implementation of the abstract exception handling.
The trycatch function should receive two states and execute the first. Should
that return an exception, it returns the result of the second state. In the
case of the abstract instance, it is also possible for a result to be either
an exception or a succesful computation. In those cases, the function should
return the join of the two results.

\subsubsection{IsBool and IsNat}
The shared interpreter needs to have a uniform way of handling values, as it
can be instantiated with either abstract or concrete values. For every type of
concrete value we will require a corresponding interface. In our case, this
means that we will require an interface for boolean types and an interface for
numerical types.

\begin{minted}{Coq}
Class IsBool (M : Type -> Type)
  (valType boolType : Type) : Type :=
{
  ensure_bool  : valType -> M boolType;
  build_bool   : boolType -> M valType;
  extract_bool : bool -> M boolType;
  and_op       : boolType -> boolType -> M boolType;
  neg_op       : boolType -> M boolType;
  if_op        : boolType -> M unit -> M unit -> M unit;
}.
\end{minted}

The IsBool typeclass is the interface for the boolean types. The interface
exposes all the operator that are possible on booleans. We've defined the and
operator and the negation operator. We also include the if operator in this
interface, as the behaviour of if will change depending on the boolean type. 

In addition, there are also the ensure, build and extract functions. The ensure
function checks whether the supplied value is actually of the type the
typeclass is instantiated for. If not, the program will crash. This ensure
function is called before the aforementioned operators, as they are not defined
for natural numbers. The build function wraps a boolean type in the value type
so that it can be passed along into other functions. The extract function can
turn a concrete value into a boolType. In the concrete implementation
of the IsBool typeclass, this will be the identity function.

\begin{minted}{Coq}
Definition ensure_abool (v : avalue) : AbstractState abstr_bool :=
  fun st => match v with
            | VAbstrBool b => returnRA b st
            | _ => crashedA
            end.

Definition eval_if_abstract {A} `{Joinable A} 
  (b : abstr_bool) (st1 st2 : AbstractState A) 
  : AbstractState A :=
  match b with
  | ab_true   => st1
  | ab_false  => st2
  | ab_top    => join_op st1 st2
  | ab_bottom => fail_abstract
  end.

Definition extract_ab (b : bool) : abstr_bool := 
  match b with
  | true => ab_true
  | false => ab_false
  end.

Instance abstract_boolean_type : IsBool AbstractState avalue abstr_bool :=
{
  ensure_bool := ensure_abool;
  extract_bool := extract_abM;
  build_bool := build_abool;
  and_op := and_abM;
  neg_op := neg_abM;
  if_op := eval_if_abstract;
}.
\end{minted}

The above listing shows the abstract implementation of the IsBool typeclass.
The abstract if evaluation takes two states and returns a state based on the
given boolean value. In the case of the Top boolean value, it returns a join of
the two states, as the abstract interpreter will be unsure of what path to
take. The listing also shows how the extract function takes a concrete value
and turns it into the correct abstraction.

\begin{minted}{Coq}
Class IsNat (M : Type -> Type)
  (valType boolType natType : Type) : Type :=
{
  ensure_nat  : valType -> M natType;
  build_nat   : natType -> M valType;
  extract_nat : nat -> M natType;
  plus_op     : natType -> natType -> M natType;
  mult_op     : natType -> natType -> M natType;
  eq_op       : natType -> natType -> M boolType;
  le_op       : natType -> natType -> M boolType;
}.
\end{minted}

The typeclass for numerical types looks much the same. It has the same ensure,
build and extract functions, as well as the operators for numerical types;
addition multiplication, equality and less-than.

Using the above interfaces, we can define the shared interpreter.
\begin{minted}{Coq}
Definition extract_build_val {M : Type -> Type} {valType boolType natType : Type}
    `{Monad M, IsNat M valType boolType natType, IsBool M valType boolType}
    (v : cvalue) : M valType :=
  match v with
  | VNat n => n' << extract_nat n; build_nat n'
  | VBool b => b' << extract_bool b; build_bool b'
  end.

Fixpoint shared_eval_expr 
    {M : Type -> Type} {valType boolType natType : Type}
    `{Monad M, Store M valType,
      IsNat M valType boolType natType, IsBool M valType boolType}
    (e : expr) : M valType :=
  match e with
  | EVal v =>
      extract_build_val v
  | EVar x =>
      get x
  | EPlus e1 e2 => 
      v1 << shared_eval_expr e1 ;
      v2 << shared_eval_expr e2 ;
      n1 << ensure_nat v1 ;
      n2 << ensure_nat v2 ;
      n << plus_op n1 n2 ;
      build_nat n
  | EMult e1 e2 => 
      v1 << shared_eval_expr e1 ;
      v2 << shared_eval_expr e2 ;
      n1 << ensure_nat v1 ;
      n2 << ensure_nat v2 ;
      n << mult_op n1 n2 ;
      build_nat n
  | EEq e1 e2 =>
      v1 << shared_eval_expr e1 ;
      v2 << shared_eval_expr e2 ;
      n1 << ensure_nat v1 ;
      n2 << ensure_nat v2 ;
      b << eq_op n1 n2 ;
      build_bool b
  | ELe e1 e2 =>
      v1 << shared_eval_expr e1 ;
      v2 << shared_eval_expr e2 ;
      n1 << ensure_nat v1 ;
      n2 << ensure_nat v2 ;
      b << le_op n1 n2;
      build_bool b
  | ENot e1 =>
      v1 << shared_eval_expr e1 ;
      b1 << ensure_bool v1 ;
      b << neg_op b1;
      build_bool b
  | EAnd e1 e2 =>
      v1 << shared_eval_expr e1 ;
      v2 << shared_eval_expr e2 ;
      b1 << ensure_bool v1 ;
      b2 << ensure_bool v2 ;
      b << and_op b1 b2 ;
      build_bool b
  end.

Fixpoint shared_ceval 
  {M : Type -> Type} {valType natType boolType : Type}
  `{Monad M, Store M valType, Except M, 
    IsNat M valType boolType natType, IsBool M valType boolType}
  (c : com) : M unit :=
  match c with
  | CSkip => returnM tt
  | c1 ;c; c2 =>
      (shared_ceval c1) ;; (shared_ceval c2)
  | x ::= a => 
      v << shared_eval_expr a ;
      put x v
  | CIf b c1 c2 => 
      v << shared_eval_expr b ;
      b' << ensure_bool v ;
      if_op b' (shared_ceval c1) (shared_ceval c2)
  | try c1 catch c2 => 
      trycatch (shared_ceval c1) (shared_ceval c2)
  | CFail => throw
  end.
\end{minted}

The required instances are added as implicited contraints to the definitions of
the shared interpreter. By doing this, we can get the type engine of Coq to
find the required instances for us. This greatly simplifies how we can now
define the abstract and concrete interpreters.

\begin{minted}{Coq}
Definition eval_expr (e : expr) : State cvalue :=
  shared_eval_expr e.

Definition ceval (c : com) : State unit :=
  shared_ceval c.
\end{minted}

We only have to define which State monad we want the interpreter to use.
Because the State type uses concrete types, it automatically infers that we
will also require the concrete instances of the Except, Store, IsBool and IsNat
typeclasses. In the next section, we shall discuss how defining the abstract
and concrete interpreters as implementations of the shared interpreter aids in
proving the soundness of the abstract interpreter.

\subsection{Soundness}\label{sec:soundness}
The main focus of this thesis is the soundness of the abstract interpreter. In
this section, we will define what it means for a abstract function to be sound
with regards to a concrete function, and show how we prove the abstract 
interpreter is sound with regards to the concrete interpreter. 

\begin{minted}{Coq}
Definition sound {A B A' B' : Type} := gamma b a.
\end{minted}

To put it in plain language, the definition of sound as defined in Coq means 
that something is sound if the gamma property holds. This goes for everything
from parities and natural numbers to two functions.

For example, \textbf{if} we have two functions $f : B \rightarrow B'$ and $f' 
: A \rightarrow A'$, where there are two Galois connections 
between $B$ and $A$, and $B'$ and $A'$ respectively, \textbf{then} when if 
there is a gamma relation between the input values, there will be a gamma
relation between the output values.

This is a bit of an obtuse definition, so we will provide some examples. Let us
look at the soundness of the parity plus function. We will show what it
means for that function to be sound with regards to standard addition of
natural numbers.

\begin{minted}{Coq}
Definition parity_plus (p q : parity) : parity :=
  match p with 
  | par_top => match q with
               | par_bottom => par_bottom
               | _ => par_top
               end
  | par_even => q
  | par_odd => match q with
               | par_top => par_top
               | par_even => par_odd
               | par_odd => par_even
               | par_bottom => par_bottom
               end
  | par_bottom => par_bottom
  end.
\end{minted}

So in more concrete terms, the function \coq{parity_plus} is sound with 
regards to \coq{plus}
if when, if the inputs are related, the output is also related. In Coq, this 
notion of relatedness is written as follows

\begin{minted}{Coq}
gamma p1 n1 -> gamma p2 n2 -> gamma (parity_plus p1 p2) (plus n1 n2).
\end{minted}

If \coq{p1} is an abstract representation of \coq{n1}, and \coq{p2} is an 
abstract representation of \coq{n2}, then \coq{parity_plus p1 p2} should be an 
abstract representation of \coq{plus n1 n2}. 

Due to the automation capabilities of Coq, the actual proof for the soundness
of the \coq{parity_plus} function is very short.

\begin{minted}{Coq}
Lemma sound_parity_plus :
  sound plus parity_plus.
Proof. intros.
  unfold sound. simpl. unfold gamma_fun; intros. simpl. 
  destruct a, a0; simpl in *; try tauto;
  auto using even_even_plus, odd_plus_r, odd_plus_l, odd_even_plus.
Qed.
\end{minted}

The proof is written using case analysis. It considers all possible
constructors of the parities and simplies the result of the \coq{parity_plus}
function. Then it uses the built-in proofs of adding combinations of even and
odd numbers to solve the subgoals.

We have similar proofs for each of the operations on values. We will not list
all those proofs here, as they are all short and of similar structure. Examine
all possible input values; resolve the functions and ascertain the preservation
of the gamma relation in those cases where the hypotheses hold.

More interesting are the soundness proofs for the interpreters. Because the
abstract and concrete interpreters both consist of the same structure, the
shared interpreter, the interpreters can be decomposed along the same lines.

We have the \coq{bind_state} and \coq{bind_state_abstract} functions that 
serve to control the flow of the program. The bind functions execute a function
and use the result of the function in the remaining functions. We can proof
that if the functions that are called are sound, and the remaining functions
are sound, the entire bind function is sound. Because the remaining functions
are also chains of bind functions, we need to proof the pairwise soundness of 
the functions to proof the entire interpreter sound.

The soundness proof for \coq{bind_state} and \coq{bind_state_abstract} is
rather long and involved, and so it omitted here for brevity. We can use the
soundness of the bind functions to decompose the soundness proof of the
interpreter. We create a custom tactic we call \coq{solve_binds}. Be repeating
the application of the soundness lemma of bind, we can turn a goal concering a
series of binds into a series of subgoals, each concering only a pair of
functions.

\begin{minted}{Coq}
Theorem eval_expr_sound : forall a,
  gamma (eval_expr_abstract a) (eval_expr a).
Proof.
  intros. unfold eval_expr_abstract, eval_expr. induction a; 
  repeat solve_binds; auto with soundness. 
Qed.

Theorem sound_interpreter:
  forall c, gamma (ceval_abstract c) (ceval c).
Proof.
  intros. unfold ceval_abstract, ceval. 
  induction c; simpl; repeat solve_binds; auto with soundness.
Qed.
\end{minted}

After we obtain a series of subgoals, we can call Coq's automation functions on
those subgoals. To do so, we need only create a hint database containing lemmas
for all the function pairs in the concrete and abstract interpreters.

\subsection{Sample Programs}
Now that we have a proof of the soundness of the abstract interpreter for every
statement in the language, we will take a look at three concrete examples of a
program written in the language. 

\begin{minted}{Coq}
Definition program1 := 
  IF2 (ELe (EVal (VNat 5)) (EVal (VNat 4))) 
  THEN (CAss "x" (EVal (VBool true))) 
  ELSE (CAss "x" (EVal (VNat 9))).

Definition program2 :=
  CAss "x" (EVal (VNat 20)) ;c;
  IF2 (EEq (EVar "x") (EVal (VNat 10)))
  THEN CFail
  ELSE (CAss "x" (EVal (VNat 20))).

Definition program3 :=
  CAss "x" 
  	(EPlus (EVal (VNat 10)) (EVal ((VBool true)))).
\end{minted}

In more conventional notation, these programs correspond to

\begin{lstlisting}
IF 5 < 4 THEN x := true ELSE x:= 9
\end{lstlisting}

\begin{lstlisting}
x := 20; IF x == 10 THEN fail ELSE x:= 20
\end{lstlisting}

\begin{lstlisting}
x := (10 + true)
\end{lstlisting}

We will discuss how the two interpreters work through these programs. When the
concrete interpreter evaluates the first program, it is able to tell that 5 is
not less than 4. As such, it evaluates the first branch of the if statement and
ignores the second. As such, it will end up with a store that contains
\textit{true} as the value of $x$.

In contrast, the abstract interpreter extracts the abstract values for 5 and 4
and will end up with an if statement that compares the set of odd numbers with
the set of even numbers. It is unable to determine which of these is smaller
than the other so it evaluates to the top value for the abstract booleans. In
this case, it joins the stores of the two branches and this results in a store
with the abstract top value for $x$. The concrete value of \textit{true} is
sound with regards to this top value, and so the abstract interpreter is sound
for the first program.

In the case of the second program, the concrete interpreter will opt for the
else branch and return a store where x holds the value of 20. The abstract
interpreter again cannot determine whether the values are equal and results in
the top abstract boolean value. It returns a state that is a join between the
failed state and a result where x is 20, using the ExceptionOrReturn
constructor. This value is a valid approximation of the concrete result, so the
interpreter is again sound.

In the third program, both the abstract and concrete interpreter exhibit the
same behaviour. Because adding a boolean to an integer is illegal, both
interpreters crash on that expression. Any concrete state is approximated by
the crashed abstract state, including a crashed concrete state, so the
interpreter is sound when analyzing this program.

